\documentclass[
    rmp, amsmath,amssymb, titlepage,
    reprint, nofootinbib, eprint
    % preprint,linenumbers
]{revtex4-2}

\usepackage[varvw]{newtx}
\usepackage{esint}
% !TEX root = ../main.tex
\usepackage[utf8]{inputenc}
\usepackage[T1]{fontenc}

\usepackage[italian]{babel}
\usepackage[babel=true]{microtype}

\usepackage{amsmath,amssymb,amsfonts}
\usepackage{siunitx}
\usepackage{physics}

\usepackage[hidelinks]{hyperref}
\usepackage[dvipsnames]{xcolor}
\usepackage{graphicx}
\usepackage[caption = false]{subfig}

\usepackage{tikz}

\usepackage[american, useregional]{datetime2}



% Hyperref setup, hidelinks in definition.
\hypersetup{
    pdfauthor={M Sotgia <s4942225@studenti.unige.it>},
    pdfcreator={Latex/UNIGE (\DTMnow)},
    pdftitle={Ok degree però short},
    pdfsubject={Shock Waves}
}

% SIUNITX
\sisetup{
    per-mode=symbol,
    separate-uncertainty=false
}

% Tikz libraries
\usetikzlibrary{calc, angles, arrows, arrows.meta, quotes}


% \renewcommand{\descriptionlabel}[1]{%
%   \hspace\labelsep \upshape\bfseries #1%
% }

\usepackage{algorithmicx, algpseudocode}

\makeatletter
\let\cat@comma@active\@empty
\makeatother


\begin{document}

\title{Metodi di Simulazione Applicati alla Fisica}
\author{Mattia Sotgia}
\email{s4942225@studenti.unige.it}

\affiliation{Università di Genova,~Dipartimento di Fisica}

\author{Riccardo Ferrando}
\email{ferrando@fisica.unige.it}

\author{Fabrizio Parodi}
\email{fabrizio.parodi@ge.infn.it}

\date{Professori del corso}
\date{\today}

\begin{abstract}
    Il corso si prefigge di fornire le conoscenze di base sulle tecniche di simulazione basate sul metodo di Monte Carlo e di applicarle alla fisica della materia e alla fisica delle interazioni fondamentali. Per la fisica della materia si acquisiranno competenze in (i) Simulazione con catene di Markov ed, in particolare, con l'algoritmo di Metropolis, (ii) Simulazione di transizioni di fase nel gas reticolare , (iii) Monte Carlo a tempo continuo per simulazione all'equilibrio e fuori equilibrio., (iv) Simulazione della crescita di aggregati. Frattali.
    
    Per la fisica delle interazioni fondamentali si acquisiranno competenze in (i) Simulazione del trasporto delle particelle nella materia, (ii) Simulazione dell'interazione e del decadimento di particelle in spazio delle fasi, (iii) Simulazione parametrica di un rivelatore, (iv) Simulazione di un esperimento composto da più rivelatori. 
\end{abstract}
\maketitle

\tableofcontents

\section{Introduzione}



\section{Markov Chain Monte Carlo}



% \bibliography{references/references}

\end{document}